\section{基本概念}
\subsection{微分方程及其解的定义}
\begin{defn}[常微分方程]\label{def:ode}
    凡是联系自变量$x$,与这个自变量的未知函数$y=y(x)$,和它的导数$y'=y'(x)$以及直到n阶导数$y^{(n)}=y^{(n)}(x)$在内的方程
    \begin{equation}\label{eq:ode}
        F(x,y,y',\ldots,y^{(n)})=0
    \end{equation}
    叫做\emphz{常微分方程},其中导数实际出现的最高阶数$n$叫做常微分方程\eqref{eq:ode}的\emphz{阶}。
\end{defn}

在常微分方程\eqref{eq:ode}中如果右端函数$F$对未知函数$y$和它的各阶导数$y',\ldots,y^{(n)}$的全体而言是一次的,则称它是\emphz{线性}常微分方程。否则称它为\emphz{非线性}常微分方程。

我们在定义\ref{def:ode}中给微分方程\eqref{eq:ode}冠以``常''字,指的是未知函数是一元函数。如果未知函数是多元函数,那么在微分方程中将出现偏导数,这种方程自然叫做\emphz{偏微分方程}。

\begin{defn}[常微分方程的解]\label{def:odesol}
    设函数$y=\varphi(x)$在区间$J$上连续,且有直到$n$阶的导数。如果把$y=\varphi(x)$及其相应的各阶导数代入方程\eqref{eq:ode},得到关于$x$的恒等式,即
    \begin{equation}\label{eq:odesol}
        F(x,\varphi(x),\varphi'(x),\ldots,\varphi^{(n)}(x))=0
    \end{equation}
    对一切$x\in J$都成立,则称$y=\varphi(x)$为微分方程\eqref{eq:ode}在区间$J$上的一个\emphz{解}。
\end{defn}

\begin{defn}[常微分方程的通解与特解]\label{def:odesol:genspec}
    设$n$阶微分方程\eqref{eq:ode}的解
    \begin{equation}\label{eq:odesol:genspec}
        y=\varphi(x,C_1,C_2,\ldots,C_n)
    \end{equation}
    包含$n$个\emphz{独立的}任意常数$C_1,C_2,\ldots,C_n$,则称它为\emphz{通解},这里所说$n$个任意常数$C_1,C_2,\ldots,C_n$是独立的,其含义是$Jacobi$行列式
    \[
        \frac{D[\varphi,\varphi',\ldots,\varphi^{(n-1)}]}{D[C_1,C_2,\ldots,C_n]}
        \overset{d}{=}
        \begin{vmatrix}
        \partialD{\varphi}{1} & \partialD{\varphi}{2} & \cdots & \partialD{\varphi}{n} \\
        \partialD{\varphi'}{1} & \partialD{\varphi'}{2} & \cdots & \partialD{\varphi'}{n} \\
        \vdots & \vdots & \ddots & \vdots \\
        \partialD{\varphi^{(n-1)}}{1} & \partialD{\varphi^{(n-1)}}{2} & \cdots & \partialD{\varphi^{(n-1)}}{n} \\
        \end{vmatrix}
    \]
    不等于0,其中
    \[
        \begin{cases}
        \varphi =\varphi(x,C_1,C_2,\ldots,C_n), & \\
        \varphi' =\varphi'(x,C_1,C_2,\ldots,C_n), & \\
        \cdots & \\
        \varphi^{(n-1)}=\varphi^{(n-1)}(x,C_1,C_2,\ldots,C_n). &
        \end{cases}
    \]
    如果微分方程\eqref{eq:ode}的解$y=\varphi(x)$不包含任意常数,则称它为\emphz{特解}。显然,当任意常数一旦确定之后,通解也就变成了特解。
\end{defn}

\begin{example}
    求双参数函数族
    \begin{equation}\label{eq:exa:doublepara}
        y=C_1e^x\cos x+C_2e^x\sin x
    \end{equation}
    所满足的微分方程。
\end{example}

事实上,在公式\eqref{eq:exa:doublepara}中对$x$先后求导两次,得出
\begin{align}
    y' &= C_1e^x(\cos x-\sin x)+C_2e^x(\sin x+\cos x)\label{eq:exa:doublepara:1} \\
    y''&= C_1e^x(-2\sin x)+C_2e^x(2\cos x)\label{eq:exa:doublepara:2}
\end{align}
从\eqref{eq:exa:doublepara:1}和\eqref{eq:exa:doublepara:2}两式可知$Jacobi$行列式
\[
    \frac{D[y,y']}{D[C_1,C_2]}
    =
    \begin{vmatrix}
    e^x\cos x & e^x\sin x \\
    e^x(\cos x-\sin x) & e^x(\sin x+\cos x)
    \end{vmatrix}
    =
    e^{2x}\neq 0.
\]
这说明\eqref{eq:exa:doublepara}中南包含的两个任意常数$C_1$和$C_2$是独立的。据此,可从\eqref{eq:exa:doublepara}和\eqref{eq:exa:doublepara:1}两式解出$C_1$和$C_2$(作为$x,y$和$y'$的函数),即
\[
    \begin{cases}
        C_1=e^{-x}[y(\sin x+\cos x)-y'\sin x], \\
        C_2=e^{-x}[y(\sin x-\cos x)+y'\cos x].
    \end{cases}
\]
然后把它们代入\eqref{eq:exa:doublepara:2}式,就得到一个二阶微分方程
\begin{equation}\label{eq:exa:doublepara:sol}
    y''-2y'+2y=0,
\end{equation}
它就是函数族\eqref{eq:exa:doublepara}所满足的微分方程;而且\eqref{eq:exa:doublepara}是微分方程\eqref{eq:exa:doublepara:sol}的通解。



\subsection{微分方程及其解的几何解释}
考虑一阶微分方程
\begin{equation}\label{eq:firstode}
    \frac{\D y}{\D x}=f(x,y),
\end{equation}
其中$f(x,y)$是平面区域$G$内的连续函数,假设
\begin{equation}\label{eq:ode:soldom}
    y=\varphi(x)\quad(x\in I)
\end{equation}
是方程的解(其中$I$是解的存在区间)。则$y=\varphi(x)$在$(x,y)$平面上的图形是一条光滑的曲线$\Gamma$,称它为微分方程\eqref{eq:firstode}的\emphz{积分曲线}。

在区域$G$内每一点$P(x,y)$,我们可以作一个以$f(P)$为斜率的(短小)直线段$l(P)$,以标明积分曲线(如果存在的话)在该店的切线方向,称$l(P)$为微分方程\eqref{eq:firstode}在$P$点的\emphz{线素};而称区域$G$联同上述全体线素为微分方程\eqref{eq:firstode}的\emphz{线素场}或\emphz{方向场}。

在构造方程\eqref{eq:firstode}的线素场时,通畅利用由关系式$f(x,y)=k$确定的曲线$L_k$,称它为线素场的\emphz{等斜线}。

一阶微分方程\eqref{eq:firstode}在许多情况取如下形式
\begin{equation}\label{eq:equform}
    \frac{\D y}{\D x}=-\dfrac{P(x,y)}{Q(x,y)},
\end{equation}
其中$P(x,y)$和$Q(x,y)$是区域$G$内的连续函数。我们可以写成下面(关于$x$和$y$)的对称形式:
\begin{equation}\label{eq:equform2}
    P(x,y)\D x+Q(x,y)\D y=0.
\end{equation}

当$P(x_0,y_0)=Q(x_0,y_0)=0$时,方程\eqref{eq:equform}在$(x_0,y_0)$点是不定式,因此线素场在$(x_0,y_0)$点没有意义。我们称这样的点$(x_0,y_0)$为相应微分方程的\emphz{奇异点}。
